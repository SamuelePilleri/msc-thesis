\thispagestyle{empty}

%% Abstract
\begin{abstract}%[\smaller \thetitle\\ \vspace*{1cm} \smaller {\theauthor}]
  %\thispagestyle{empty}
  Nell'ultimo decennio il settore delle telecomunicazioni ha visto la nascita di molteplici iniziative con il comune obiettivo di creare un ecosistema hardware e software aperto per le reti Internet, superando il modello chiuso proposto dai Vendor tradizionali. In questo scenario, il software riveste un ruolo chiave: dal 5G all'edge computing, le nuove architetture devono essere flessibili, scalabili ed economiche. Si inizia quindi a parlare di \textit{telco cloud}, dove le prestazioni e il \mbox{management} evoluto delle odierne piattaforme di virtualizzazione, basate su hypervisor e container, permettono di implementare in maniera efficace e flessibile le funzioni di rete prima vincolate al deploy di appliance fisiche.

  Lo scopo di questa tesi è in primis fornire una panoramica sullo stack tecnologico hardware e software necessario per realizzare l'architettura, descrivendone lo stato dell'arte, le varie componenti in gioco e come si integrano tra loro. In secondo luogo, si propone un'architettura virtuale che realizzi una particolare funzione, il \mbox{peering} decentralizzato all'edge della rete, studiandone le performance attraverso i principali indicatori.
\end{abstract}


%% Declaration
\begin{declaration}
  Dichiaro di essere responsabile del contenuto dell'elaborato che presento al fine del
  conseguimento del titolo, di non avere plagiato in tutto o in parte il lavoro prodotto da
  altri e di aver citato le fonti originali in modo congruente alle normative vigenti in
  materia di plagio e di diritto d'autore. Sono inoltre consapevole che nel caso la mia
  dichiarazione risultasse mendace, potrei incorrere nelle sanzioni previste dalla legge e la
  mia ammissione alla prova finale potrebbe essere negata.
  \vspace*{1cm}
  \begin{flushright}
    Samuele Pilleri
  \end{flushright}
\end{declaration}


% % Acknowledgements
% \begin{acknowledgements}
%   Ringrazio i miei familiari ed i loro sforzi che mi hanno permesso, pur non senza intoppi, di raggiungere questa tanto agognata meta.
  
%   Ringrazio i professori, presenti e passati, le cui lezioni continueranno ad arricchirmi anche dopo la fine della scuola.
  
%   Ringrazio i miei amici, anche quelli del baretto, per avermi ispirato, incoraggiato, sopportato e per i momenti di leggerezza vissuti insieme.
  
%   Ma il grazie più sentito va al mio mentore, Alessandro, conosciuto per caso in un gruppo Telegram: a te sono grato di avermi spalancato le porte delle telco, accogliendomi nel piccolo giardino del TOP-IX e facendomi sentire subito a casa. Mi hai guidato nel muovere i primi passi verso il raggiante futuro che ci aspetta, senza mai essere avaro di sapere. Quelli trascorsi in via delle Rosine sono stati i sei mesi più ricchi da quando mi sono immatricolato, non lo dimenticherò mai.
  
%   Grazie a tutti voi questo lavoro è stato svolto non solo con passione, ma con gioia!
% \end{acknowledgements}


%% Preface
% \begin{preface}
%   This thesis describes my research on various aspects of the \LHCb
%   particle physics program, centred around the \LHCb detector and \LHC
%   accelerator at \CERN in Geneva.

%   \noindent
%   For this example, I'll just mention \ChapterRef{chap:SomeStuff}
%   and \ChapterRef{chap:MoreStuff}.
% \end{preface}

%% ToC
\tableofcontents


%% Strictly optional!
\frontquote{%
  Non chi comincia, ma quel che persevera.}%
  {Leonardo da Vinci}
%% I don't want a page number on the following blank page either.
\thispagestyle{empty}
